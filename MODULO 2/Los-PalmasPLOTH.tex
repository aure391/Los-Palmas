% Options for packages loaded elsewhere
\PassOptionsToPackage{unicode}{hyperref}
\PassOptionsToPackage{hyphens}{url}
%
\documentclass[
]{article}
\usepackage{amsmath,amssymb}
\usepackage{iftex}
\ifPDFTeX
  \usepackage[T1]{fontenc}
  \usepackage[utf8]{inputenc}
  \usepackage{textcomp} % provide euro and other symbols
\else % if luatex or xetex
  \usepackage{unicode-math} % this also loads fontspec
  \defaultfontfeatures{Scale=MatchLowercase}
  \defaultfontfeatures[\rmfamily]{Ligatures=TeX,Scale=1}
\fi
\usepackage{lmodern}
\ifPDFTeX\else
  % xetex/luatex font selection
\fi
% Use upquote if available, for straight quotes in verbatim environments
\IfFileExists{upquote.sty}{\usepackage{upquote}}{}
\IfFileExists{microtype.sty}{% use microtype if available
  \usepackage[]{microtype}
  \UseMicrotypeSet[protrusion]{basicmath} % disable protrusion for tt fonts
}{}
\makeatletter
\@ifundefined{KOMAClassName}{% if non-KOMA class
  \IfFileExists{parskip.sty}{%
    \usepackage{parskip}
  }{% else
    \setlength{\parindent}{0pt}
    \setlength{\parskip}{6pt plus 2pt minus 1pt}}
}{% if KOMA class
  \KOMAoptions{parskip=half}}
\makeatother
\usepackage{xcolor}
\usepackage[margin=1in]{geometry}
\usepackage{color}
\usepackage{fancyvrb}
\newcommand{\VerbBar}{|}
\newcommand{\VERB}{\Verb[commandchars=\\\{\}]}
\DefineVerbatimEnvironment{Highlighting}{Verbatim}{commandchars=\\\{\}}
% Add ',fontsize=\small' for more characters per line
\usepackage{framed}
\definecolor{shadecolor}{RGB}{248,248,248}
\newenvironment{Shaded}{\begin{snugshade}}{\end{snugshade}}
\newcommand{\AlertTok}[1]{\textcolor[rgb]{0.94,0.16,0.16}{#1}}
\newcommand{\AnnotationTok}[1]{\textcolor[rgb]{0.56,0.35,0.01}{\textbf{\textit{#1}}}}
\newcommand{\AttributeTok}[1]{\textcolor[rgb]{0.13,0.29,0.53}{#1}}
\newcommand{\BaseNTok}[1]{\textcolor[rgb]{0.00,0.00,0.81}{#1}}
\newcommand{\BuiltInTok}[1]{#1}
\newcommand{\CharTok}[1]{\textcolor[rgb]{0.31,0.60,0.02}{#1}}
\newcommand{\CommentTok}[1]{\textcolor[rgb]{0.56,0.35,0.01}{\textit{#1}}}
\newcommand{\CommentVarTok}[1]{\textcolor[rgb]{0.56,0.35,0.01}{\textbf{\textit{#1}}}}
\newcommand{\ConstantTok}[1]{\textcolor[rgb]{0.56,0.35,0.01}{#1}}
\newcommand{\ControlFlowTok}[1]{\textcolor[rgb]{0.13,0.29,0.53}{\textbf{#1}}}
\newcommand{\DataTypeTok}[1]{\textcolor[rgb]{0.13,0.29,0.53}{#1}}
\newcommand{\DecValTok}[1]{\textcolor[rgb]{0.00,0.00,0.81}{#1}}
\newcommand{\DocumentationTok}[1]{\textcolor[rgb]{0.56,0.35,0.01}{\textbf{\textit{#1}}}}
\newcommand{\ErrorTok}[1]{\textcolor[rgb]{0.64,0.00,0.00}{\textbf{#1}}}
\newcommand{\ExtensionTok}[1]{#1}
\newcommand{\FloatTok}[1]{\textcolor[rgb]{0.00,0.00,0.81}{#1}}
\newcommand{\FunctionTok}[1]{\textcolor[rgb]{0.13,0.29,0.53}{\textbf{#1}}}
\newcommand{\ImportTok}[1]{#1}
\newcommand{\InformationTok}[1]{\textcolor[rgb]{0.56,0.35,0.01}{\textbf{\textit{#1}}}}
\newcommand{\KeywordTok}[1]{\textcolor[rgb]{0.13,0.29,0.53}{\textbf{#1}}}
\newcommand{\NormalTok}[1]{#1}
\newcommand{\OperatorTok}[1]{\textcolor[rgb]{0.81,0.36,0.00}{\textbf{#1}}}
\newcommand{\OtherTok}[1]{\textcolor[rgb]{0.56,0.35,0.01}{#1}}
\newcommand{\PreprocessorTok}[1]{\textcolor[rgb]{0.56,0.35,0.01}{\textit{#1}}}
\newcommand{\RegionMarkerTok}[1]{#1}
\newcommand{\SpecialCharTok}[1]{\textcolor[rgb]{0.81,0.36,0.00}{\textbf{#1}}}
\newcommand{\SpecialStringTok}[1]{\textcolor[rgb]{0.31,0.60,0.02}{#1}}
\newcommand{\StringTok}[1]{\textcolor[rgb]{0.31,0.60,0.02}{#1}}
\newcommand{\VariableTok}[1]{\textcolor[rgb]{0.00,0.00,0.00}{#1}}
\newcommand{\VerbatimStringTok}[1]{\textcolor[rgb]{0.31,0.60,0.02}{#1}}
\newcommand{\WarningTok}[1]{\textcolor[rgb]{0.56,0.35,0.01}{\textbf{\textit{#1}}}}
\usepackage{graphicx}
\makeatletter
\def\maxwidth{\ifdim\Gin@nat@width>\linewidth\linewidth\else\Gin@nat@width\fi}
\def\maxheight{\ifdim\Gin@nat@height>\textheight\textheight\else\Gin@nat@height\fi}
\makeatother
% Scale images if necessary, so that they will not overflow the page
% margins by default, and it is still possible to overwrite the defaults
% using explicit options in \includegraphics[width, height, ...]{}
\setkeys{Gin}{width=\maxwidth,height=\maxheight,keepaspectratio}
% Set default figure placement to htbp
\makeatletter
\def\fps@figure{htbp}
\makeatother
\setlength{\emergencystretch}{3em} % prevent overfull lines
\providecommand{\tightlist}{%
  \setlength{\itemsep}{0pt}\setlength{\parskip}{0pt}}
\setcounter{secnumdepth}{-\maxdimen} % remove section numbering
\ifLuaTeX
  \usepackage{selnolig}  % disable illegal ligatures
\fi
\IfFileExists{bookmark.sty}{\usepackage{bookmark}}{\usepackage{hyperref}}
\IfFileExists{xurl.sty}{\usepackage{xurl}}{} % add URL line breaks if available
\urlstyle{same}
\hypersetup{
  pdftitle={Primeros pasos en R},
  pdfauthor={Los-Palmas},
  hidelinks,
  pdfcreator={LaTeX via pandoc}}

\title{Primeros pasos en R}
\author{Los-Palmas}
\date{2023-03-29}

\begin{document}
\maketitle

\hypertarget{r-markdown}{%
\subsection{R Markdown}\label{r-markdown}}

This is an R Markdown document. Markdown is a simple formatting syntax
for authoring HTML, PDF, and MS Word documents. For more details on
using R Markdown see \url{http://rmarkdown.rstudio.com}.

When you click the \textbf{Knit} button a document will be generated
that includes both content as well as the output of any embedded R code
chunks within the document. You can embed an R code chunk like this:

\begin{Shaded}
\begin{Highlighting}[]
\FunctionTok{summary}\NormalTok{(cars)}
\end{Highlighting}
\end{Shaded}

\begin{verbatim}
##      speed           dist       
##  Min.   : 4.0   Min.   :  2.00  
##  1st Qu.:12.0   1st Qu.: 26.00  
##  Median :15.0   Median : 36.00  
##  Mean   :15.4   Mean   : 42.98  
##  3rd Qu.:19.0   3rd Qu.: 56.00  
##  Max.   :25.0   Max.   :120.00
\end{verbatim}

\hypertarget{including-plots}{%
\subsection{Including Plots}\label{including-plots}}

You can also embed plots, for example:

\includegraphics{Los-PalmasPLOTH_files/figure-latex/pressure-1.pdf}

Note that the \texttt{echo\ =\ FALSE} parameter was added to the code
chunk to prevent printing of the R code that generated the plot.

\hypertarget{definiciuxf3n-de-variables-numuxe9ricas}{%
\subsection{Definición de variables
numéricas}\label{definiciuxf3n-de-variables-numuxe9ricas}}

para definir variables numericas basta con darle un nombre a la variable
y poner la flecha de asignación seguida del valor numérico que queremos
asignar. en la ventana superior derecha aparecerá la variable con su
valor

\begin{Shaded}
\begin{Highlighting}[]
\NormalTok{ca}\OtherTok{\textless{}{-}}\DecValTok{16}
\NormalTok{ca}
\end{Highlighting}
\end{Shaded}

\begin{verbatim}
## [1] 16
\end{verbatim}

\hypertarget{vectores}{%
\subsubsection{VECTORES}\label{vectores}}

Los vectores son otra estructura de datos de R que a diferencia de las
variables que vimos antes, alojan una colección de valores. Desde el
punto de vista de objetos una variable simple es un vector que solo
tiene un elemento.

\begin{Shaded}
\begin{Highlighting}[]
\NormalTok{Ventas }\OtherTok{\textless{}{-}} \FunctionTok{c}\NormalTok{ (}\DecValTok{30}\NormalTok{,}\DecValTok{35}\NormalTok{,}\DecValTok{23}\NormalTok{,}\DecValTok{45}\NormalTok{,}\DecValTok{60}\NormalTok{,}\DecValTok{69}\NormalTok{,}\DecValTok{12}\NormalTok{,}\DecValTok{34}\NormalTok{,}\DecValTok{36}\NormalTok{,}\DecValTok{89}\NormalTok{,}\DecValTok{74}\NormalTok{,}\DecValTok{25}\NormalTok{)}
\FunctionTok{plot}\NormalTok{(Ventas)}
\end{Highlighting}
\end{Shaded}

\includegraphics{Los-PalmasPLOTH_files/figure-latex/unnamed-chunk-2-1.pdf}

Quisiera saber cuál es el nivel promedio de ventas anuales y el desvío
estandar.

\begin{Shaded}
\begin{Highlighting}[]
\FunctionTok{mean}\NormalTok{ (Ventas)}
\end{Highlighting}
\end{Shaded}

\begin{verbatim}
## [1] 44.33333
\end{verbatim}

\begin{Shaded}
\begin{Highlighting}[]
\FunctionTok{sd}\NormalTok{ (Ventas)}
\end{Highlighting}
\end{Shaded}

\begin{verbatim}
## [1] 23.49597
\end{verbatim}

\#\#\#DATOS SIMULADOS

Queremos generar unos datos para meter en un simulador que representen a
nuestra empresa. Vamos a generar 500 datos con el mismo promedio y
desvío estándar.

\begin{Shaded}
\begin{Highlighting}[]
\NormalTok{Ventas\_simuladas}\OtherTok{\textless{}{-}}\FunctionTok{rnorm}\NormalTok{(}\DecValTok{500}\NormalTok{,}\FloatTok{44.33333}\NormalTok{,}\FloatTok{23.49597}\NormalTok{)}
\FunctionTok{plot}\NormalTok{(Ventas\_simuladas)}
\end{Highlighting}
\end{Shaded}

\includegraphics{Los-PalmasPLOTH_files/figure-latex/unnamed-chunk-4-1.pdf}

Qué probabilidad tengo de que las ventas sean menores a 30

\begin{Shaded}
\begin{Highlighting}[]
\FunctionTok{pnorm}\NormalTok{(}\DecValTok{30}\NormalTok{,}\FloatTok{44.33}\NormalTok{,}\FloatTok{23.49}\NormalTok{)}
\end{Highlighting}
\end{Shaded}

\begin{verbatim}
## [1] 0.2709154
\end{verbatim}

\hypertarget{variables-de-tipo-lista}{%
\subsection{VARIABLES DE TIPO LISTA}\label{variables-de-tipo-lista}}

Las variables de tipo lista son vectores pero que se generan con el
comando ``scan()''. Como scan es un comando interactivo no podemos
colocarlo dentro de un documento Rmarkdown. Sólo podemos ejecutar en la
ventana de consola el comando.

\begin{verbatim}
compras <-scan()
\end{verbatim}

\#\#Comando length y seq length: funciona para evidenciar la longitud de
datos con la que cuenta una variable definida anteriormente. seq: se
utiliza para generar una secuencia de números con un valor inicial, un
valor final y el rango entre valores que se van dando. Mediante estas
funciones se puede generar una gráfica de un cierto valor sin necesidad
de contar con dos datos de la siguiente manera:

\begin{Shaded}
\begin{Highlighting}[]
\FunctionTok{plot}\NormalTok{(}\FunctionTok{seq}\NormalTok{(}\DecValTok{1}\NormalTok{,}\FunctionTok{length}\NormalTok{(Ventas\_simuladas),}\DecValTok{1}\NormalTok{),Ventas\_simuladas)}
\end{Highlighting}
\end{Shaded}

\includegraphics{Los-PalmasPLOTH_files/figure-latex/unnamed-chunk-6-1.pdf}

\hypertarget{muxe9todos-alternativos-de-generar-secuencias}{%
\subsection{MÉTODOS ALTERNATIVOS DE GENERAR
SECUENCIAS}\label{muxe9todos-alternativos-de-generar-secuencias}}

A los programadores les encanta usar los métodos de programación que
llevan comandos como ``for'', ``until'', ``while'' ; pero estos métodos
son caros computacionalmente hablando. Como R trabaja con comando
matricial podes hacerlo solo en una linea de comando.

\begin{Shaded}
\begin{Highlighting}[]
\NormalTok{impuestos}\OtherTok{\textless{}{-}}\DecValTok{0}
\NormalTok{impuestos[}\DecValTok{1}\NormalTok{] }\OtherTok{\textless{}{-}} \DecValTok{0}
\ControlFlowTok{for}\NormalTok{ (i }\ControlFlowTok{in} \DecValTok{2}\SpecialCharTok{:}\DecValTok{24}\NormalTok{) \{}
\NormalTok{impuestos[i] }\OtherTok{\textless{}{-}}\NormalTok{ impuestos[i}\DecValTok{{-}1}\NormalTok{]}\SpecialCharTok{+}\DecValTok{2}\SpecialCharTok{*}\NormalTok{i}
\NormalTok{impuestos}
\NormalTok{\}}
\NormalTok{impuestos}
\end{Highlighting}
\end{Shaded}

\begin{verbatim}
##  [1]   0   4  10  18  28  40  54  70  88 108 130 154 180 208 238 270 304 340 378
## [20] 418 460 504 550 598
\end{verbatim}

De la lista ventas\_simuladas ¿Cuántos valores son superiores a 40?

\begin{Shaded}
\begin{Highlighting}[]
\NormalTok{indice }\OtherTok{\textless{}{-}} \FunctionTok{which}\NormalTok{(Ventas\_simuladas }\SpecialCharTok{\textgreater{}} \DecValTok{40}\NormalTok{)}
\NormalTok{indice}
\end{Highlighting}
\end{Shaded}

\begin{verbatim}
##   [1]   2   3   4   6   7   8   9  10  12  13  14  17  20  21  23  25  27  30
##  [19]  31  34  35  37  38  39  40  41  42  43  45  47  50  53  54  57  58  64
##  [37]  65  68  70  71  73  76  77  79  80  83  85  88  89  91  92  93  95  96
##  [55]  98  99 100 101 102 104 105 106 107 108 109 110 111 113 114 115 116 118
##  [73] 120 121 123 124 125 128 129 131 133 134 137 138 139 143 145 147 148 149
##  [91] 150 151 152 154 156 157 161 164 169 172 174 175 177 178 180 182 185 186
## [109] 187 189 191 193 195 196 198 201 204 205 208 210 215 216 217 219 220 222
## [127] 226 230 231 232 233 235 236 238 239 242 243 244 246 248 249 250 251 252
## [145] 253 255 256 260 261 263 264 266 268 269 272 273 276 277 278 279 280 282
## [163] 284 285 287 288 289 291 294 295 296 298 299 300 301 302 303 304 307 308
## [181] 309 310 312 313 314 318 319 322 325 326 328 329 330 331 332 335 336 337
## [199] 338 339 340 343 348 351 356 357 358 360 361 362 365 366 369 370 371 373
## [217] 374 375 376 378 380 382 383 384 385 388 389 392 393 394 397 398 399 400
## [235] 401 402 403 405 406 407 409 411 412 413 414 416 417 421 422 423 424 425
## [253] 428 429 430 433 437 438 440 444 445 446 447 450 452 456 457 459 462 464
## [271] 466 467 470 471 477 478 480 481 482 483 484 485 486 489 491 492 494 495
## [289] 496 497 499
\end{verbatim}

\begin{Shaded}
\begin{Highlighting}[]
\NormalTok{indice }\OtherTok{\textless{}{-}}  \FunctionTok{which}\NormalTok{(Ventas\_simuladas }\SpecialCharTok{\textgreater{}} \DecValTok{40}\NormalTok{)}
\NormalTok{indice}
\end{Highlighting}
\end{Shaded}

\begin{verbatim}
##   [1]   2   3   4   6   7   8   9  10  12  13  14  17  20  21  23  25  27  30
##  [19]  31  34  35  37  38  39  40  41  42  43  45  47  50  53  54  57  58  64
##  [37]  65  68  70  71  73  76  77  79  80  83  85  88  89  91  92  93  95  96
##  [55]  98  99 100 101 102 104 105 106 107 108 109 110 111 113 114 115 116 118
##  [73] 120 121 123 124 125 128 129 131 133 134 137 138 139 143 145 147 148 149
##  [91] 150 151 152 154 156 157 161 164 169 172 174 175 177 178 180 182 185 186
## [109] 187 189 191 193 195 196 198 201 204 205 208 210 215 216 217 219 220 222
## [127] 226 230 231 232 233 235 236 238 239 242 243 244 246 248 249 250 251 252
## [145] 253 255 256 260 261 263 264 266 268 269 272 273 276 277 278 279 280 282
## [163] 284 285 287 288 289 291 294 295 296 298 299 300 301 302 303 304 307 308
## [181] 309 310 312 313 314 318 319 322 325 326 328 329 330 331 332 335 336 337
## [199] 338 339 340 343 348 351 356 357 358 360 361 362 365 366 369 370 371 373
## [217] 374 375 376 378 380 382 383 384 385 388 389 392 393 394 397 398 399 400
## [235] 401 402 403 405 406 407 409 411 412 413 414 416 417 421 422 423 424 425
## [253] 428 429 430 433 437 438 440 444 445 446 447 450 452 456 457 459 462 464
## [271] 466 467 470 471 477 478 480 481 482 483 484 485 486 489 491 492 494 495
## [289] 496 497 499
\end{verbatim}

\begin{Shaded}
\begin{Highlighting}[]
\NormalTok{Ventas\_simuladas[indice]}
\end{Highlighting}
\end{Shaded}

\begin{verbatim}
##   [1]  49.03910  64.03950  72.36563  60.40400  99.67228  70.30053  57.06992
##   [8]  65.08394  47.58757  61.28439  47.62881  96.96016  44.94309  55.09291
##  [15]  52.83025  51.08420  45.13749  56.57086  58.09544  51.81319  60.30724
##  [22]  77.40142  52.99937  51.11671  45.32485  68.09366  47.18671  59.97186
##  [29]  74.60788  43.88176  44.51196  53.96511  65.55747  50.33782  49.00724
##  [36]  56.99126  53.16141  77.21299  54.29932  60.63171 103.76666  52.54411
##  [43]  77.63030  48.22020  56.67844  58.24951  44.63282  77.27746  58.89197
##  [50]  79.54254  83.07300  63.84462  40.82292  42.82186  57.97400  63.93546
##  [57]  84.86298  43.66817  71.97135  57.48228  70.43936  85.20919  84.32743
##  [64]  50.91797  72.35003  51.04420  67.09964  55.48624  49.88521  48.14437
##  [71]  47.10004  67.10119  73.04599  44.55598  51.34424  44.34273  74.54958
##  [78]  54.82446  52.62785  53.07358  42.08793  46.70206  46.47173  40.30826
##  [85]  47.80418  41.38541  49.29202  60.78396  43.48379  85.64381  66.60159
##  [92]  74.73292  43.11935  55.79106  69.56880  47.44792  65.34523  60.16684
##  [99]  50.72538  72.58764  42.00286 103.91838  95.60405  47.44172  52.83964
## [106]  56.31623  45.22693  67.15204  98.42285  85.54203  70.81080  55.25458
## [113]  85.13805  66.07872  90.75882  64.06030  66.60078  67.40641  49.72802
## [120]  55.68885  52.58110  45.11409  65.40367  64.43528  42.60786  43.88158
## [127]  49.38643  41.41036  74.78464  86.65565 100.51755  50.22092  40.87705
## [134]  48.95430  44.13254  89.29377  61.93059  68.22012  87.05964  41.65248
## [141]  71.33065  42.16367  66.93891  50.88484  73.94314  55.00733  97.98297
## [148]  72.77153  68.96659  52.92353  61.61635  90.47186  41.32587  63.21692
## [155]  43.42964  40.98273  55.67823  89.26316  65.01488  84.45359  92.03825
## [162]  47.52069  62.67999  42.85993  51.49836  52.36321  45.61507  58.74281
## [169]  53.11155  46.36954  54.44257  42.92744  64.78453  60.74758  54.98883
## [176]  76.10830  41.77403  45.42361  43.65465  59.88680  76.87722  67.93899
## [183]  48.90611  82.34848  54.02921  47.85170  45.08970  42.35371  53.30889
## [190]  47.87377  68.98079  65.75158  60.54956  73.35119  51.22429  41.18597
## [197]  82.88052  73.10843  51.07147  48.69236  46.15154  45.37539  79.18089
## [204]  40.37693  60.36723  48.86912  76.04880  48.46449  57.14665  60.89938
## [211]  65.85140  49.97132  48.19948  64.12835  74.41681  52.35215  57.41997
## [218]  45.71325  44.34828  80.40724  58.39588  61.58097  60.01189  65.45727
## [225]  78.26355  46.78077  60.99363  57.39556  46.12530  43.82342  42.51271
## [232]  53.43937  61.44229  50.95524  46.88712  89.46310  45.21613  49.34436
## [239]  88.83435  41.04753  72.90732  51.45847  57.71478  40.83112  98.89082
## [246]  49.80236  45.02253  64.94427  87.63875  47.07130  49.24389  70.40785
## [253]  81.02074  76.43416  69.63942  41.47268  96.71474  49.10205  69.29121
## [260]  69.15555  60.33870  82.54667  83.24379  42.63990  56.86669  40.66754
## [267]  40.86200  95.52514  68.97142  76.13798  50.51751  54.40246  58.75427
## [274]  44.97210  46.28758  51.04146  67.44378  63.77530  81.88655  40.66198
## [281]  71.64687  49.57504  79.31314  81.13146  49.39857  75.74170  47.55430
## [288]  54.17938  57.74720  47.25044  62.65431
\end{verbatim}

\begin{Shaded}
\begin{Highlighting}[]
\FunctionTok{length}\NormalTok{(indice)}
\end{Highlighting}
\end{Shaded}

\begin{verbatim}
## [1] 291
\end{verbatim}

\begin{Shaded}
\begin{Highlighting}[]
\FunctionTok{sum}\NormalTok{(Ventas\_simuladas[indice])}
\end{Highlighting}
\end{Shaded}

\begin{verbatim}
## [1] 17529.53
\end{verbatim}

\begin{Shaded}
\begin{Highlighting}[]
\NormalTok{A }\OtherTok{\textless{}{-}} \DecValTok{0} 
\NormalTok{ A[}\DecValTok{1}\NormalTok{] }\OtherTok{\textless{}{-}} \DecValTok{0} 
\NormalTok{ A[}\DecValTok{2}\NormalTok{] }\OtherTok{\textless{}{-}} \DecValTok{1}
\NormalTok{ A[}\DecValTok{3}\NormalTok{] }\OtherTok{\textless{}{-}}\NormalTok{ A[}\DecValTok{1}\NormalTok{]}\SpecialCharTok{+}\NormalTok{A[}\DecValTok{2}\NormalTok{]}
 \ControlFlowTok{for}\NormalTok{ (i }\ControlFlowTok{in} \DecValTok{3}\SpecialCharTok{:}\DecValTok{31}\NormalTok{) \{A[i] }\OtherTok{\textless{}{-}}\NormalTok{ A[i}\DecValTok{{-}1}\NormalTok{]}\SpecialCharTok{+}\NormalTok{A[i}\DecValTok{{-}2}\NormalTok{]\}}
\NormalTok{A}
\end{Highlighting}
\end{Shaded}

\begin{verbatim}
##  [1]      0      1      1      2      3      5      8     13     21     34
## [11]     55     89    144    233    377    610    987   1597   2584   4181
## [21]   6765  10946  17711  28657  46368  75025 121393 196418 317811 514229
## [31] 832040
\end{verbatim}

\end{document}
